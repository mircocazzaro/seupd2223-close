\documentclass{ceurart}

\usepackage[footnote]{acronym}
\usepackage{subcaption}
\usepackage{dirtree}
\usepackage{listings}
\usepackage{float}
\usepackage{lmodern}
\usepackage{graphicx}
\usepackage{wrapfig}
\usepackage{lscape}
\usepackage{rotating}
\usepackage{epstopdf}
\usepackage[export]{adjustbox}[2011/08/13]


\restylefloat{table}



\lstset{frame=tb,
  language=Java,
  aboveskip=3mm,
  belowskip=3mm,
  showstringspaces=false,
  columns=flexible,
  basicstyle={\small\ttfamily},
  numbers=none,
  breaklines=true,
  breakatwhitespace=true,
  tabsize=3
}

\renewcommand{\acffont}[1]{\textsl{#1}}


%%
%% end of the preamble, start of the body of the document source.
\begin{document}

%%
%% Rights management information.
%% CC-BY is default license.
\copyrightyear{2023}
\copyrightclause{Copyright for this paper by its authors.\\
  Use permitted under Creative Commons License Attribution 4.0
  International (CC BY 4.0).}

%%
%% This command is for the conference information
\conference{``Search Engines'', course at the master degree in ``Computer Engineering'', Department of Information Engineering, and at the master degree in ``Data Science'', Department of Mathematics ``Tullio Levi-Civita'', University of Padua, Italy. Academic Year 2022/2023}

%%
%% The "title" command
\title{SEUPD@CLEF: Team CLOSE}

%%
%% The "author" command and its associated commands are used to define
%% the authors and their affiliations.
\author[1]{Antolini Gianluca}[%
email=gianluca.antolini@studenti.unipd.it
]

\author[1]{Boscolo Cegion Nicola}[%
email=nicola.boscolocegion.1@studenti.unipd.it
]

\author[1]{Cazzaro Mirco}[%
email=mirco.cazzaro@studenti.unipd.it
]

\author[1]{Martinelli Marco}[%
email=marco.martinelli.4@studenti.unipd.it
]

\author[1]{Safavi Seyedreza}[%
email=seyedreza.safavi@studenti.unipd.it
]

\author[1]{Shami Farzad}[%
email=farzad.shami@studenti.unipd.it
]

\address[1]{University of Padua, Italy}


%%
%% The abstract is a short summary of the work to be presented in the
%% article.
\begin{abstract}
  This paper presents the work of the \textit{CLOSE} group, a team of students from the University of Padua, Italy, for the \ac{CLEF} LongEval LAB 2023 Task 1~\cite{cleflongeval}.
  Our work involved developing an Information Retrieval system that can handle changes in data over time while maintaining high performance.
  We first introduce the problem as stated by \ac{CLEF} and then describe our system, explaining the different methodologies we implemented.
  We provide the results of our experiments and analyze them based on the choices we made regarding various techniques.
  Finally, we propose potential avenues for future improvement of our system.
\end{abstract}

%%
%% Keywords. The author(s) should pick words that accurately describe
%% the work being presented. Separate the keywords with commas.
\begin{keywords}
  Information Retrieval \sep
  Search Engines \sep
  Longitudinal Evaluation \sep
  Model Performance
\end{keywords}

%%
%% This command processes the author and affiliation and title
%% information and builds the first part of the formatted document.
\maketitle


\enlargethispage{2\baselineskip}
\section{Introduction}
\label{sec:introduction}

Recent research has shown that the performance of information retrieval systems can deteriorate over time as the data they are trained on becomes less relevant to current search queries. 
This problem is particularly acute when dealing with temporal information, as web documents and user search preferences evolve over time. 
In this paper, we propose a solution to this problem by developing an information retrieval system that can adapt to changes in the data over time, while maintaining high performance.

Our approach involves using the training data provided by the \textit{Qwant}~\cite{qwant} search engine, which includes user searches and web documents in both French and English.
We believe that this data will enable our system to better adapt to changes in user search behavior and the content of web documents.

The remainder of this paper is organized as follows: 
Section~\ref{sec:methodology} describes our approach in more detail, including the different techniques we used. 
Section~\ref{sec:setup} explains our experimental setup, including the datasets and evaluation metrics we used. 
Section~\ref{sec:results} presents our main findings and analyzes them based on the choices we made regarding various techniques. 
Section~\ref{sec:analysis} goes into deep in analyzing results of our systems over test data and within time performances, making use of statistical tools such as ANOVA.
Finally, Section~\ref{sec:conclusion} summarizes our conclusions and outlines potential avenues for future work.

\section{Methodology}
\label{sec:methodology}

Describe the methodology you have adopted, the architecture of your system, your workflow, etc.

\newpage
\enlargethispage{3\baselineskip}
\section{Experimental Setup}
\label{sec:setup}

\subsection{Collections}
We developed our model using a collection of 1,593,376 documents and 882 queries provided by \textit{Qwant} search engine, available at \url{https://lindat.mff.cuni.cz/repository/xmlui/handle/11234/1-5010}.
\newline
The collection contains information about user web searches and actual web pages corpora. The data was originally all in French but, for both queries and documents, an English translation is provided.


\subsection{Evaluation Measures}
To measure the effectiveness of our \ac{IR} system we used the \textit{trec\_eval} executable by testing it with the resulting runs produced by the model and its different configurations.
\newline
We tracked improvements of the following evaluation measures generated by \textit{trec\_eval}:
\begin{itemize}
	\item \textbf{num\textunderscore ret}: number of documents retrieved for a given query.
	\item \textbf{num\textunderscore rel}: number of relevant documents for a given query.
	\item \textbf{num \textunderscore rel\textunderscore ret}: number of relevant documents retrieved for a given query.
    \item \textbf{map}: Mean Average Precision, a measure of the average relevance of retrieved documents across all queries. 
    \item \textbf{rprec}: R-Precision is the precision score computed at the rank corresponding to the number of relevant documents for a given query.
    \item \textbf{p@5} and \textbf{p@10}: Precision at 5 and at 10 is the precision computed at the top 5 and 10 retrieved documents for a given query.
    \item \textbf{nDCG}: it is a metric used to evaluate ranked lists. It measures the effectiveness of a ranking algorithm by considering item relevance.
\end{itemize}


\subsection{Git Repository}
More detailed information about our information retrieval system, including source code, runs, results, homework reports, and presentation slides, can be found in our Git repository at \url{https://bitbucket.org/upd-dei-stud-prj/seupd2223-close/src/master/%7D}. 
The code is available for reproducibility.

\pagebreak
\section{Results and Discussion}

\label{sec:results}

%\begin{table}[h!]
%    \centering
%    \begin{tabular}{ |l|c|c|c|c|c| }
%        \hline
%        metrics & run1 & run2 & run3 & run4 & run5 \\ \hline
%        num\_q & 657 & 669 & 669 & 667 & 667 \\ \hline
%        num\_ret & 646525 & 658446 & 658347 & 652222 & 657903 \\ \hline
%        num\_rel & 2550 & 2611 & 2611 & 2603 & 2600 \\ \hline
%        num\_rel\_ret & 1772 & 2182 & 2191 & 1866 & 2232 \\ \hline
%        map & 0.1307 & 0.2022 & 0.2335 & 0.1856 & 0.2351 \\ \hline
%        gm\_map & 0.0117 & 0.046 & 0.061 & 0.0239 & 0.0629 \\ \hline
%        Rprec & 0.1041 & 0.1697 & 0.1989 & 0.1654 & 0.2022 \\ \hline
%        bpref & 0.3142 & 0.3734 & 0.3869 & 0.3466 & 0.3861 \\ \hline
%        recip\_rank & 0.2436 & 0.3287 & 0.3891 & 0.3441 & 0.3945 \\ \hline
%        iprec\_at\_recall\_0.00 & 0.2553 & 0.3499 & 0.4134 & 0.3584 & 0.4182 \\ \hline
%        iprec\_at\_recall\_0.20 & 0.2387 & 0.3324 & 0.3873 & 0.3353 & 0.3927 \\ \hline
%        iprec\_at\_recall\_0.40 & 0.1441 & 0.2316 & 0.2732 & 0.2066 & 0.2716 \\ \hline
%        iprec\_at\_recall\_0.60 & 0.0965 & 0.1786 & 0.1996 & 0.1388 & 0.2014 \\ \hline
%        iprec\_at\_recall\_0.80 & 0.0628 & 0.1178 & 0.1311 & 0.0887 & 0.1295 \\ \hline
%        iprec\_at\_recall\_1.00 & 0.0525 & 0.0954 & 0.1031 & 0.0704 & 0.1028 \\ \hline
%        P\_10 & 0.0848 & 0.1296 & 0.1435 & 0.1126 & 0.1432 \\ \hline
%        P\_100 & 0.0186 & 0.0256 & 0.0268 & 0.0222 & 0.0268 \\ \hline
%        P\_1000 & 0.0027 & 0.0033 & 0.0033 & 0.0028 & 0.0033 \\ \hline
%        recall\_10 & 0.2166 & 0.3352 & 0.367 & 0.2849 & 0.3621 \\ \hline
%        recall\_100 & 0.4718 & 0.6426 & 0.6714 & 0.5536 & 0.6723 \\ \hline
%        recall\_1000 & 0.6816 & 0.8192 & 0.8218 & 0.7004 & 0.8392 \\ \hline
%        infAP & 0.1307 & 0.2022 & 0.2335 & 0.1856 & 0.2351 \\ \hline
%        gm\_bpref & 0.0152 & 0.0387 & 0.0405 & 0.022 & 0.038 \\ \hline
%        utility & -978.6621 & -977.701 & -977.5262 & -972.2489 & -979.6687 \\ \hline
%        ndcg & 0.2719 & 0.3655 & 0.3924 & 0.3291 & 0.3982 \\ \hline
%        ndcg\_rel & 0.236 & 0.3119 & 0.3416 & 0.2939 & 0.3471 \\ \hline
%        Rndcg & 0.1708 & 0.2387 & 0.2657 & 0.2271 & 0.2714 \\ \hline
%        ndcg\_cut\_5 & 0.1285 & 0.1908 & 0.2232 & 0.1854 & 0.2269 \\ \hline
%        ndcg\_cut\_10 & 0.1609 & 0.2426 & 0.2739 & 0.2227 & 0.2758 \\ \hline
%        ndcg\_cut\_100 & 0.2351 & 0.3349 & 0.3652 & 0.3016 & 0.3678 \\ \hline
%        ndcg\_cut\_1000 & 0.2719 & 0.3655 & 0.3924 & 0.3291 & 0.3982 \\ \hline
%        map\_cut\_10 & 0.1046 & 0.1665 & 0.1975 & 0.1556 & 0.1993 \\ \hline
%        map\_cut\_100 & 0.1284 & 0.2 & 0.2315 & 0.1836 & 0.2328 \\ \hline
%        map\_cut\_1000 & 0.1307 & 0.2022 & 0.2335 & 0.1856 & 0.2351 \\ \hline
%
%    \end{tabular}
%    \caption{English results from TREC eval}
%    \label{table:results}
%\end{table}

%\begin{itemize}
%    \item \textbf{run1}: PorterStemFilter, Standard tokenizer, LenghtFilter between 2 and 15, "long-stoplist-fr.txt" for StopFilter, LowerCaseFilter, BM25Similarity;
%    \item \textbf{run2}: FrenchLightStemFilter, Standard tokenizer, LenghtFilter between 2 and 15, "long-stoplist-fr.txt" for StopFilter, LowerCaseFilter, BM25Similarity, Query Expansion;
%    \item \textbf{run3}: FrenchLightStemFilter, Standard tokenizer, LenghtFilter between 2 and 15, "long-stoplist-fr.txt" for StopFilter, LowerCaseFilter,\\ BM25Similarity((float)1.2,(float)0.90), Query Expansion, Re-ranking;
%    \item \textbf{run4}: PorterStemFilter, Standard tokenizer, LenghtFilter between 2 and 15, "long-stoplist.txt" for StopFilter, LowerCaseFilter, BM25Similarity((float)1.2,(float)0.90), EnglishPossessiveFilter, Query Expansion, Re-ranking;
%    \item \textbf{run5}: FrenchLightStemFilter, Standard tokenizer, LenghtFilter between 2 and 15, "new-long-stoplist-fr.txt" for StopFilter, LowerCaseFilter,\\ BM25Similarity((float)1.2,(float)0.90), ElisionFilter with some common french articles and prepositions, Query Expansion, Re-ranking;
%\end{itemize}

\begin{table}
    \hspace*{-2.4cm}
    \begin{tabularx}{1.33\textwidth}{|l|X|X|X|X|X|} 
    \hline
    \textbf{Parameter} & \textbf{Run 1} & \textbf{Run 2} & \textbf{Run 3} & \textbf{Run 4} & \textbf{Run 5} \\
    \hline
    Token Filter & Porter- \newline StemFilter & FrenchLight- \newline StemFilter & FrenchLight- \newline StemFilter & PorterStem- \newline Filter & FrenchLight- \newline StemFilter \\
    \hline
    Tokenizer & Standard & Standard & Standard & Standard & Standard \\
    \hline
    Length Filter & 2-15 & 2-15 & 2-15 & 2-15 & 2-15 \\
    \hline
    Stop Filter & "\textit{long-stoplist.txt}" & "\textit{long-stoplist-fr.txt}" & "\textit{long-stoplist-fr.txt}" & "\textit{long-stoplist.txt}" & "\textit{new-long-stoplist-fr.txt}" \\
    \hline
    Lower Case Filter & Yes & Yes & Yes & Yes & Yes \\
    \hline
    Similarity & BM25 & BM25 & BM25 & BM25 & BM25 \\
    \hline
    Query Expansion & No & Yes & Yes & Yes & Yes \\
    \hline
    Re-ranking & No & No & Yes & Yes & Yes \\
    \hline
    \end{tabularx}
    \caption{Parameters used in the 5 different runs submitted to \ac{CLEF}}
    \label{tab:run_parameters}
\end{table}

\begin{table}[h!]
    \centering
    \begin{tabular}{ |l|c|c|c|c|c| }
        \hline
        metrics & run1 & run2 & run3 & run4 & run5 \\ \hline
        num\_q & 657 & 669 & 669 & 667 & 667 \\ \hline
        num\_ret & 646525 & 658446 & 658347 & 652222 & 657903 \\ \hline
        num\_rel & 2550 & 2611 & 2611 & 2603 & 2600 \\ \hline
        num\_rel\_ret & 1772 & 2182 & 2191 & 1866 & 2232 \\ \hline
        map & 0.1307 & 0.2022 & 0.2335 & 0.1856 & 0.2351 \\ \hline
        Rprec & 0.1041 & 0.1697 & 0.1989 & 0.1654 & 0.2022 \\ \hline
        iprec\_at\_recall\_0.00 & 0.2553 & 0.3499 & 0.4134 & 0.3584 & 0.4182 \\ \hline
        iprec\_at\_recall\_0.20 & 0.2387 & 0.3324 & 0.3873 & 0.3353 & 0.3927 \\ \hline
        iprec\_at\_recall\_0.40 & 0.1441 & 0.2316 & 0.2732 & 0.2066 & 0.2716 \\ \hline
        iprec\_at\_recall\_0.60 & 0.0965 & 0.1786 & 0.1996 & 0.1388 & 0.2014 \\ \hline
        iprec\_at\_recall\_0.80 & 0.0628 & 0.1178 & 0.1311 & 0.0887 & 0.1295 \\ \hline
        iprec\_at\_recall\_1.00 & 0.0525 & 0.0954 & 0.1031 & 0.0704 & 0.1028 \\ \hline
        P\_10 & 0.0848 & 0.1296 & 0.1435 & 0.1126 & 0.1432 \\ \hline
        P\_100 & 0.0186 & 0.0256 & 0.0268 & 0.0222 & 0.0268 \\ \hline
        P\_1000 & 0.0027 & 0.0033 & 0.0033 & 0.0028 & 0.0033 \\ \hline
        recall\_10 & 0.2166 & 0.3352 & 0.367 & 0.2849 & 0.3621 \\ \hline
        recall\_100 & 0.4718 & 0.6426 & 0.6714 & 0.5536 & 0.6723 \\ \hline
        recall\_1000 & 0.6816 & 0.8192 & 0.8218 & 0.7004 & 0.8392 \\ \hline
        ndcg & 0.2719 & 0.3655 & 0.3924 & 0.3291 & 0.3982 \\ \hline
        ndcg\_cut\_5 & 0.1285 & 0.1908 & 0.2232 & 0.1854 & 0.2269 \\ \hline
        ndcg\_cut\_10 & 0.1609 & 0.2426 & 0.2739 & 0.2227 & 0.2758 \\ \hline
        ndcg\_cut\_100 & 0.2351 & 0.3349 & 0.3652 & 0.3016 & 0.3678 \\ \hline
        ndcg\_cut\_1000 & 0.2719 & 0.3655 & 0.3924 & 0.3291 & 0.3982 \\ \hline
    \end{tabular}
    \caption{results for systems (top-1000 documents), on the Train collection and the train query set of LongEval.}
    \label{table:results}
\end{table}

%\begin{figure}[h!]
%    \includegraphics[width=\textwidth]{figure/PRgraph.png}
%    \caption{Recall and precision graph}
%    \label{fig:recallPrecision}
%  \end{figure}
In this Section, we provide some of the most relevant results we got during the development phase.
We are considering five principal milestones that, within many different trials, led us to improve significantly our \ac{MAP} score and the overall number of relevant documents actually retrieved.
\subsection{Results on Training Data}
First of all, given that we were provided with two different versions of the same document's \textit{corpora}, our first idea was to try the English version.\\
We noticed that the best combination of basic \ac{IR} tools was to use the \textit{Porter} Stemmer \cite{solrporterstemfilter}, a length filter from 1 to 10, and a list of stop-word composed by some standard terms and more from the top 600 extracted from the index.
The very first big milestone, that helped us to increment the \ac{MAP} of around 3 points, from 10.1\% to 13.1\%, was the \textit{JavaScript} code cleaner since we noticed by inspection that many documents were having these types of scripts inside.\\
Always by inspecting some documents and queries, and also considering that the original collection was the French version (translated then in English), we observed that the translation was very poor: by switching to French
by just cleaning the \textit{JS} code and some other minor cleaning tools, without even using an adequate stop-list and a correct stemmer for the French language, the \ac{MAP} was increasing by +5\%. \\
2 more \ac{MAP} points were achieved with a stop list built for French in the same way we did previously for English, the \textit{FrenchLightStemFilter}~\cite{solrfrenchlightstemfilter} as stemmer, and moving the length filter from 2 to 15 (as we noticed French tends to have longer words). \\
We tried some \ac{NLP} techniques for English to see if there were improvements, and in this case, apply them to our main implementation for French with an appropriate model. The obtained results were not interesting, and also the computing time was definitely too costly. In particular, we tried to use Solr OpenNLP Part of Speech Filter \cite{solropennlpposfilter} using the \textit{en-pos-maxent} \ac{PoS} tagger provided by \textit{OpenNLP}.
Another approach we tried and that carried an improvement was to use \textit{Query expansion}: first we used some generative text models to expand our queries, then we decided to weight different query scores by boosting the original one linearly with respect to the number
%of expansion used, and without boosting the expansion: this carried to us an extra MAP point.
of expansion used. This made us gain an extra \ac{MAP} point.

\newpage
\enlargethispage{4\baselineskip}
\begin{figure}[h!]
    \includegraphics[width=\textwidth]{figure/PRgraph.png}
    \caption{Standard Recall Levels vs Interpolated Precision}
    \label{fig:recallPrecision}
\end{figure}

We try to generate the embeddings for each document based on word2vector, we use a pre-trained word2vec model \textit{frWac\_no\_postag\_no\_phrase\_500\_cbow\_cut100}~\cite{fauconnier_2015} for French. Then we calculate the embedding for each document and index them as KnnFloatVectorField in Lucene and use KnnFloatVectorQuery \cite{lucene-knnvectorfield} for searching the query to find the \textit{k} nearest documents to the target vector according to the vectors in the given field, but the results (overall \ac{MAP} ~0.08) were not satisfying, being worse than the case of indexing and searching without embeddings.
\newline
We then tried to combine different similarities rather than using the classic \textit{BM25Similarity}: we tried to use the Lucene \textit{MultiSimilarity} \cite{lucenemultisimilarity}, that allows combining the score of two or more similarity scores, but it does not allow to
tune the weights. Then, we tried to reimplement the \textit{MultiSimilarity} class with tuning options, but the results were always lower than the standard \textit{BM25Similarity}. Some minor improvements came up by fine-tuning the document-length
normalization \textit{b} parameter and the term frequency component \textit{k1} parameter of the \textit{BM25Similarity}.
\newline
The last main implementation we did, was to use some \textit{Re-ranking} techniques to improve the results of the first retrieval phase.
We tried to use the \textit{SBERT} model~\cite{reimers-2019-sentence-bert}, which is a pre-trained model for sentence embeddings, and we used it to calculate the similarity between the query and the document.
We tried to use different distance metrics such as \textit{CosineSimilarity} \cite{pytorch-cosinesimilarity} and \textit{ManhattanDistance} \cite{dads-manhattandistance} for calculating the similarity, but at the end of the day, \textit{CosineSimilarity} is much better than others.
Finally, we sort the documents based on merging the BM25 score and similarity score into one score by multiplying them together.

Lastly, some minor adding were set on the \textit{Analyzer} (see Section \ref{analyzer_subsec}) by implementing the Lucene \textit{ElisionFilter} (for French) \cite{luceneelisionfilter}, which aims to remove apostrophes articles and prepositions from tokens (for example, \textit{m'appelle} and \textit{t'appelle} become the same token \textit{appelle}).


\subsection{Results on Test Data}
\label{subsec:results_submission}
%\begin{table}[h!]
%    \centering
%    \begin{tabular}{ |c|c|c|c|c|c|c| }
%        \hline
%        \hline
%        \multicolumn{7}{|c|}{heldout} \\ \hline
%        run  & language & type & map & p@10 & NDCG & recall \\ \hline
%        run2 &   FR   & QUEREXPANSION & 0.2029 & 0.1367 & 0.3725 & 0.8312 \\
%        run3 &   FR   & RERANKING & 0.2595 & 0.1541 & 0.4166 & 0.8348 \\
%        run5 &   FR   & SBERT\_BM25 & 0.2675 & 0.1561 & 0.4318 & 0.8726 \\
%        \hline
%        run1 &   EN   & JSCLEANER\_BM25 & 0.1299 & 0.0897 & 0.2674 & 0.6381 \\
%        run4 &   EN   & RERANKING\_ENGLISH & 0.1822 & 0.1122 & 0.3113 & 0.6279 \\
%        \hline
%    \end{tabular}
%    \caption{results for systems (top-1000 documents), on the Train collection and the heldout query set of LongEval.}
%    \label{tab:results_submission_heldout}
%\end{table}
%
%\begin{table}[h!]
%    \centering
%    \begin{tabular}{ |c|c|c|c|c|c|c| }
%        \hline
%        \hline
%        \multicolumn{7}{|c|}{Short term} \\ \hline
%        run & language & type & map & p@10 & NDCG & recall \\ \hline
%        run2 &   FR   & QUEREXPANSION & 0.2215 & 0.1326 & 0.3800 & 0.8164 \\
%        run3 &   FR   & RERANKING & 0.2511 & 0.2171 & 0.4073 & 0.8142 \\
%        run5 &   FR   & SBERT\_BM25 & 0.2540 & 0.1497 & 0.4142 & 0.8360 \\
%        \hline
%        run1 &   EN   & JSCLEANER\_BM25 & 0.1438 & 0.0902 & 0.2746 & 0.6566 \\
%        run4 &   EN   & RERANKING\_ENGLISH & 0.1956 & 0.1145 & 0.3311 & 0.6804 \\
%        \hline
%        \hline
%        \multicolumn{7}{|c|}{Long term} \\ \hline
%        run  & language & type & map & p@10 & NDCG & recall \\ \hline
%        run2 &   FR   & QUEREXPANSION & 0.2067 & 0.1423 & 0.3745 & 0.8312 \\
%        run3 &   FR   & RERANKING & 0.2388 & 0.1555 & 0.4071 & 0.8336 \\
%        run5 &   FR   & SBERT\_BM25 & 0.2437 & 0.1594 & 0.4148 & 0.8540 \\
%        \hline
%        run1 &   EN   & JSCLEANER\_BM25  & 0.1450 & 0.0975 & 0.2866 & 0.6906 \\
%        run4 &   EN   & RERANKING\_ENGLISH & 0.1930 & 0.1258 & 0.3391 & 0.7119 \\
%        \hline
%    \end{tabular}
%    \caption{results for systems (top-1000 documents), on the Test collection and the test query set of LongEval.}
%    \label{tab:results_submission_test}
%\end{table}

\begin{table}[h!]
    \centering
    \begin{tabular}{ |c|c|c|c|c|c|c|c| }
        \hline
        \hline
        \multicolumn{8}{|c|}{heldout} \\ \hline
        run  & language & type & map & p@10 & recall & nDCG & nDCG@10 \\ \hline
        run2 &   FR   & QUEREXPANSION & 0.2029 & 0.1367 & 0.8312 & 0.3725 & 0.2436 \\
        run3 &   FR   & RERANKING & 0.2595 & 0.1541 & 0.8384 & 0.4166 & 0.2925 \\
        run5 &   FR   & SBERT\_BM25 & 0.2675 & 0.1561 & 0.8726 & 0.4318 & 0.3017 \\
        \hline
        run1 &   EN   & JSCLEANER\_BM25 & 0.1282 & 0.0888 & 0.6316 & 0.2647 & 0.1565 \\
        run4 &   EN   & RERANKING\_ENGLISH & 0.1822 & 0.1122 & 0.6279 & 0.3113 & 0.2129 \\
        \hline
    \end{tabular}
    \caption{results for systems (top-1000 documents), on the Train collection and the heldout query set of LongEval provied by LongEval.}
    \label{tab:results_submission_heldout}
\end{table}

\begin{table}[h!]
    \centering
    \begin{tabular}{ |c|c|c|c|c|c|c|c| }
        \hline
        \hline
        \multicolumn{8}{|c|}{Short term} \\ \hline
        run  & language & type & map & p@10 & recall & nDCG & nDCG@10 \\ \hline
        run2 &   FR   & QUEREXPANSION & 0.2213 & 0.1324 & 0.8155 & 0.3795 & 0.2583 \\
        run3 &   FR   & RERANKING & 0.2508 & 0.1483 & 0.8133 & 0.4068 & 2944 \\
        run5 &   FR   & SBERT\_BM25 & 0.2531 & 0.1492 & 0.8332 & 0.4128 & 0.2963 \\
        \hline
        run1 &   EN   & JSCLEANER\_BM25 & 0.1410 & 0.0884 & 0.6440 & 0.2694 & 0.1683 \\
        run4 &   EN   & RERANKING\_ENGLISH & 0.1941 & 0.1136 & 0.6750 & 0.3285 & 0.2303 \\
        \hline
        \hline
        \multicolumn{8}{|c|}{Long term} \\ \hline
        run  & language & type & map & p@10 & recall & nDCG & nDCG@10 \\ \hline
        run2 &   FR   & QUEREXPANSION & 0.2062 & 0.1420 & 0.8294 & 0.3736 & 0.2438 \\
        run3 &   FR   & RERANKING & 0.2383 & 0.1551 & 0.8318 & 0.4062 & 0.2821 \\
        run5 &   FR   & SBERT\_BM25 & 0.2432 & 0.1590 & 0.8521 & 0.4139 & 0.2880 \\
        \hline
        run1 &   EN   & JSCLEANER\_BM25 & 0.1419 & 0.0953 & 0.6756 & 0.2803 & 0.1672 \\
        run4 &   EN   & RERANKING\_ENGLISH & 0.1920 & 0.1251 & 0.7080 & 0.3373 & 0.2275 \\
        \hline
    \end{tabular}
    \caption{results for systems (top-1000 documents), on the Test collection and the test query set of LongEval provied by LongEval.}
    \label{tab:results_submission_test}
\end{table}


\pagebreak


\newpage
\section{Conclusions and Future Work}
\label{sec:conclusion}
 
In this work, we presented our approach to the \ac{CLEF} \textit{Long Eval LAB 2023} task, which aimed to develop an effective and efficient search engine for web documents. \\
Our approach consisted of using a combination of different techniques, including query expansion, re-ranking, and the use of large language models such as \textit{ChatGPT} and \textit{SBERT}. \\
Our experiments showed that our approach achieved good results in terms of effectiveness and efficiency, outperforming the baseline system provided by \ac{CLEF}. Specifically, we found that combining two different scores in the re-ranking phase led to significant improvements in the retrieval performance. Moreover, we identified several areas for future work that could further improve the effectiveness and efficiency of our approach. \\
One possible direction for future work is to find better ways to combine scores or add other scores to the re-ranking phase. We plan to explore different combinations of scores and investigate the use of other large language models, such as other available \textit{BERT} models trained, or to train some specifically for this task. \\
Another area for future work is to find better prompts \cite{wang2023chatgpt} to use in \textit{ChatGPT} for improving query expansion. We also plan to investigate the use of other \ac{LLM} techniques for query expansion. \\
We also want to explore ways to increase the similarity in \textit{SBERT}~\cite{reimers-2019-sentence-bert}, to increase the number of relevant documents found in the re-ranking phase. One possible approach is to fine-tune the \textit{SBERT}~\cite{reimers-2019-sentence-bert} model on our specific task. \\
Another direction for future work is to index documents as vectors and use them directly, instead of calculating them in re-ranking. This trade-off would result in the loss of one of the scores, but it would increase the re-ranking speed. \\
Finally, we plan to use links inside documents to extract details that may improve the searching results. We may try to find keywords in the URL path and use them to find their domain authority and take this aspect into account in the score computation.

\newpage




%% Define the bibliography file to be used
\bibliography{bibliography,proceedings}

\acrodef{3G}[3G]{Third Generation Mobile System}
\acrodef{5S}[5S]{Streams, Structures, Spaces, Scenarios, Societies}
\acrodef{AA}[AA]{Active Agreements}
\acrodef{AAAI}[AAAI]{Association for the Advancement of Artificial Intelligence}
\acrodef{AAL}[AAL]{Annotation Abstraction Layer}
\acrodef{AAM}[AAM]{Automatic Annotation Manager}
\acrodef{AAP}[AAP]{Average Average Precision}
\acrodef{ACLIA}[ACLIA]{Advanced Cross-Lingual Information Access}
\acrodef{ACM}[ACM]{Association for Computing Machinery}
\acrodef{AD}[AD]{Active Disagreements}
\acrodef{ADSL}[ADSL]{Asymmetric Digital Subscriber Line}
\acrodef{ADUI}[ADUI]{ADministrator User Interface}
\acrodef{AIP}[AIP]{Archival Information Package}
\acrodef{AJAX}[AJAX]{Asynchronous JavaScript Technology and \acs{XML}}
\acrodef{ALU}[ALU]{Aritmetic-Logic Unit}
\acrodef{AMUSID}[AMUSID]{Adaptive MUSeological IDentity-service}
\acrodef{ANOVA}[ANOVA]{ANalysis Of VAriance}
\acrodef{ANSI}[ANSI]{American National Standards Institute}
\acrodef{AP}[AP]{Average Precision}
\acrodef{APC}[APC]{AP Correlation}
\acrodef{API}[API]{Application Program Interface}
\acrodef{AR}[AR]{Address Register}
\acrodef{AS}[AS]{Annotation Service}
\acrodef{ASAP}[ASAP]{Adaptable Software Architecture Performance}
\acrodef{ASI}[ASI]{Annotation Service Integrator}
\acrodef{ASL}[ASL]{Achieved Significance Level}
\acrodef{ASM}[ASM]{Annotation Storing Manager}
\acrodef{ASR}[ASR]{Automatic Speech Recognition}
\acrodef{ASUI}[ASUI]{ASsessor User Interface}
\acrodef{ATIM}[ATIM]{Annotation Textual Indexing Manager}
\acrodef{AUC}[AUC]{Area Under the ROC Curve}
\acrodef{AUI}[AUI]{Administrative User Interface}
\acrodef{AWARE}[AWARE]{Assessor-driven Weighted Averages for Retrieval Evaluation}
\acrodef{BANKS-I}[BANKS-I]{Browsing ANd Keyword Searching I}
\acrodef{BANKS-II}[BANKS-II]{Browsing ANd Keyword Searching II}
\acrodef{BH}[BH]{Benjamini-Hochberg}
\acrodef{bpref}[bpref]{Binary Preference}
\acrodef{BNF}[BNF]{Backus and Naur Form}
\acrodef{BPM}[BPM]{Bejeweled Player Model}
\acrodef{BRICKS}[BRICKS]{Building Resources for Integrated Cultural Knowledge Services}
\acrodef{CAN}[CAN]{Content Addressable Netword}
\acrodef{CAS}[CAS]{Content-And-Structure}
\acrodef{CBSD}[CBSD]{Component-Based Software Developlement}
\acrodef{CBSE}[CBSE]{Component-Based Software Engineering}
\acrodef{CB-SPE}[CB-SPE]{Component-Based \acs{SPE}}
\acrodef{CD}[CD]{Collaboration Diagram}
\acrodef{CD}[CD]{Compact Disk}
\acrodef{CDF}[CDF]{Cumulative Density Function}
\acrodef{CENL}[CENL]{Conference of European National Librarians}
\acrodef{CIDOC CRM}[CIDOC CRM]{CIDOC Conceptual Reference Model}
\acrodef{CIR}[CIR]{Current Instruction Register}
\acrodef{CIRCO}[CIRCO]{Coordinated Information Retrieval Components Orchestration}
\acrodef{CG}[CG]{Cumulated Gain}
\acrodef{CL}[CL]{Curriculum Learning}
\acrodef{CL-ESA}[CL-ESA]{Cross-Lingual Explicit Semantic Analysis}
\acrodef{CLAIRE}[CLAIRE]{Combinatorial visuaL Analytics system for Information Retrieval Evaluation}
\acrodef{CLEF1}[CLEF]{Cross-Language Evaluation Forum}
\acrodef{CLEF}[CLEF]{Conference and Labs of the Evaluation Forum}
\acrodef{CLIR}[CLIR]{Cross Language Information Retrieval}
\acrodef{CM}[CM]{Continuation Methods}
\acrodef{CMS}[CMS]{Content Management System}
\acrodef{CMT}[CMT]{Campaign Management Tool}
\acrodef{CNR}[CNR]{Italian National Council of Research}
\acrodef{CO}[CO]{Content-Only}
\acrodef{COD}[COD]{Code On Demand}
\acrodef{CODATA}[CODATA]{Committee on Data for Science and Technology}
\acrodef{COLLATE}[COLLATE]{Collaboratory for Annotation Indexing and Retrieval of Digitized Historical Archive Material}
\acrodef{CP}[CP]{Characteristic Pattern}
\acrodef{CPE}[CPE]{Control Processor Element}
\acrodef{CPU}[CPU]{Central Processing Unit}
\acrodef{CQL}[CQL]{Contextual Query Language}
\acrodef{CRP}[CRP]{Cumulated Relative Position}
\acrodef{CRUD}[CRUD]{Create--Read--Update--Delete}
\acrodef{CS}[CS]{Characteristic Structure}
\acrodef{CSM}[CSM]{Campaign Storing Manager}
\acrodef{CSS}[CSS]{Cascading Style Sheets}
\acrodef{CTR}[CTR]{Click-Through Rate}
\acrodef{CU}[CU]{Control Unit}
\acrodef{CUI}[CUI]{Client User Interface}
\acrodef{CV}[CV]{Cross-Validation}
\acrodef{DAFFODIL}[DAFFODIL]{Distributed Agents for User-Friendly Access of Digital Libraries}
\acrodef{DAO}[DAO]{Data Access Object}
\acrodef{DARE}[DARE]{Drawing Adequate REpresentations}
\acrodef{DARPA}[DARPA]{Defense Advanced Research Projects Agency}
\acrodef{DAS}[DAS]{Distributed Annotation System}
\acrodef{DB}[DB]{DataBase}
\acrodef{DBMS}[DBMS]{DataBase Management System}
\acrodef{DC}[DC]{Dublin Core}
\acrodef{DCG}[DCG]{Discounted Cumulated Gain}
\acrodef{DCMI}[DCMI]{Dublin Core Metadata Initiative}
\acrodef{DCV}[DCV]{Document Cut--off Value}
\acrodef{DD}[DD]{Deployment Diagram}
\acrodef{DDC}[DDC]{Dewey Decimal Classification}
\acrodef{DDS}[DDS]{Direct Data Structure}
\acrodef{DF}[DF]{Degrees of Freedom}
\acrodef{DFI}[DFI]{Divergence From Independence}
\acrodef{DFR}[DFR]{Divergence From Randomness}
\acrodef{DHT}[DHT]{Distributed Hash Table}
\acrodef{DI}[DI]{Digital Image}
\acrodef{DIKW}[DIKW]{Data, Information, Knowledge, Wisdom}
\acrodef{DIL}[DIL]{\acs{DIRECT} Integration Layer}
\acrodef{DiLAS}[DiLAS]{Digital Library Annotation Service}
\acrodef{DIRECT}[DIRECT]{Distributed Information Retrieval Evaluation Campaign Tool}
\acrodef{DKMS}[DKMS]{Data and Knowledge Management System}
\acrodef{DL}[DL]{Digital Library}
\acrodefplural{DL}[DL]{Digital Libraries}
\acrodef{DLMS}[DLMS]{Digital Library Management System}
\acrodef{DLOG}[DL]{Description Logics}
\acrodef{DLS}[DLS]{Digital Library System}
\acrodef{DLSS}[DLSS]{Digital Library Service System}
\acrodef{DM}[DM]{Data Mining}
\acrodef{DO}[DO]{Digital Object}
\acrodef{DOI}[DOI]{Digital Object Identifier}
\acrodef{DOM}[DOM]{Document Object Model}
\acrodef{DoMDL}[DoMDL]{Document Model for Digital Libraries}
\acrodef{DP}[DP]{Discriminative Power}
\acrodef{DPBF}[DPBF]{Dynamic Programming Best-First}
\acrodef{DR}[DR]{Data Register}
\acrodef{DRIVER}[DRIVER]{Digital Repository Infrastructure Vision for European Research}
\acrodef{DTD}[DTD]{Document Type Definition}
\acrodef{DVD}[DVD]{Digital Versatile Disk}
\acrodef{EAC-CPF}[EAC-CPF]{Encoded Archival Context for Corporate Bodies, Persons, and Families}
\acrodef{EAD}[EAD]{Encoded Archival Description}
\acrodef{EAN}[EAN]{International Article Number}
\acrodef{EBU}[EBU]{Expected Browsing Utility}
\acrodef{ECD}[ECD]{Enhanced Contenty Delivery}
\acrodef{ECDL}[ECDL]{European Conference on Research and Advanced Technology for Digital Libraries}
\acrodef{EDM}[EDM]{Europeana Data Model}
\acrodef{EG}[EG]{Execution Graph}
\acrodef{ELDA}[ELDA]{Evaluation and Language resources Distribution Agency}
\acrodef{ELRA}[ELRA]{European Language Resources Association}
\acrodef{EM}[EM]{Expectation Maximization}
\acrodef{EMMA}[EMMA]{Extensible MultiModal Annotation}
\acrodef{EPROM}[EPROM]{Erasable Programmable \acs{ROM}}
\acrodef{EQNM}[EQNM]{Extended Queueing Network Model}
\acrodef{ER}[ER]{Entity--Relationship}
\acrodef{ERR}[ERR]{Expected Reciprocal Rank}
\acrodef{ERS}[ERS]{Empirical Relational System}
\acrodef{ESA}[ESA]{Explicit Semantic Analysis}
\acrodef{ESL}[ESL]{Expected Search Length}
\acrodef{ETL}[ETL]{Extract-Transform-Load}
\acrodef{FAST}[FAST]{Flexible Annotation Service Tool}
\acrodef{FDR}[FDR]{False Discovery Rate}
\acrodef{FIFO}[FIFO]{First-In / First-Out}
\acrodef{FIRE}[FIRE]{Forum for Information Retrieval Evaluation}
\acrodef{FN}[FN]{False Negative}
\acrodef{FNR}[FNR]{False Negative Rate}
\acrodef{FOAF}[FOAF]{Friend of a Friend}
\acrodef{FORESEE}[FORESEE]{FOod REcommentation sErvER}
\acrodef{FP}[FP]{False Positive}
\acrodef{FPR}[FPR]{False Positive Rate}
\acrodef{FWER}[FWER]{Family-wise Error Rate}
\acrodef{GIF}[GIF]{Graphics Interchange Format}
\acrodef{GIR}[GIR]{Geografic Information Retrieval}
\acrodef{GAP}[GAP]{Graded Average Precision}
\acrodef{GLM}[GLM]{General Linear Model}
\acrodef{GLMM}[GLMM]{General Linear Mixed Model}
\acrodef{GMAP}[GMAP]{Geometric Mean Average Precision}
\acrodef{GoP}[GoP]{Grid of Points}
\acrodef{GPRS}[GPRS]{General Packet Radio Service}
\acrodef{gP}[gP]{Generalized Precision}
\acrodef{gR}[gR]{Generalized Recall}
\acrodef{gRBP}[gRBP]{Graded Rank-Biased Precision}
\acrodef{GT}[GT]{Generalizability Theory}
\acrodef{GTIN}[GTIN]{Global Trade Item Number}
\acrodef{GUI}[GUI]{Graphical User Interface}
\acrodef{GW}[GW]{Gateway}
\acrodef{HCI}[HCI]{Human Computer Interaction}
\acrodef{HDS}[HDS]{Hybrid Data Structure}
\acrodef{HIR}[HIR]{Hypertext Information Retrieval}
\acrodef{HIT}[HIT]{Human Intelligent Task}
\acrodef{HITS}[HITS]{Hyperlink-Induced Topic Search}
\acrodef{HMM}[HMM]{Hidden Markov Model}
\acrodef{HTML}[HTML]{HyperText Markup Language}
\acrodef{HTTP}[HTTP]{HyperText Transfer Protocol}
\acrodef{HSD}[HSD]{Honestly Significant Difference}
\acrodef{ICA}[ICA]{International Council on Archives}
\acrodef{ICSU}[ICSU]{International Council for Science}
\acrodef{IDF}[IDF]{Inverse Document Frequency}
\acrodef{IDS}[IDS]{Inverse Data Structure}
\acrodef{IEEE}[IEEE]{Institute of Electrical and Electronics Engineers}
\acrodef{IEI}[IEI]{Istituto della Enciclopedia Italiana fondata da Giovanni Treccani}
\acrodef{IETF}[IETF]{Internet Engineering Task Force}
\acrodef{IIR}[IIR]{Interactive Information Retrieval}
\acrodef{IMS}[IMS]{Information Management System}
\acrodef{IMSPD}[IMS]{Information Management Systems Research Group}
\acrodef{indAP}[indAP]{Induced Average Precision}
\acrodef{infAP}[infAP]{Inferred Average Precision}
\acrodef{INEX}[INEX]{INitiative for the Evaluation of \acs{XML} Retrieval}
\acrodef{INS-M}[INS-M]{Inverse Set Data Model}
\acrodef{INTR}[INTR]{Interrupt Register}
\acrodef{IP}[IP]{Internet Protocol}
\acrodef{IPSA}[IPSA]{Imaginum Patavinae Scientiae Archivum}
\acrodef{IR}[IR]{Information Retrieval}
\acrodef{IRON}[IRON]{Information Retrieval ON}
\acrodef{IRON2}[IRON$^2$]{Information Retrieval On aNNotations}
\acrodef{IRON-SAT}[IRON-SAT]{\acs{IRON} - Statistical Analysis Tool}
\acrodef{IRS}[IRS]{Information Retrieval System}
\acrodef{ISAD(G)}[ISAD(G)]{International Standard for Archival Description (General)}
\acrodef{ISBN}[ISBN]{International Standard Book Number}
\acrodef{ISIS}[ISIS]{Interactive SImilarity Search}
\acrodef{ISJ}[ISJ]{Interactive Searching and Judging}
\acrodef{ISO}[ISO]{International Organization for Standardization}
\acrodef{ITU}[ITU]{International Telecommunication Union }
\acrodef{ITU-T}[ITU-T]{Telecommunication Standardization Sector of \acs{ITU}}
\acrodef{IV}[IV]{Information Visualization}
\acrodef{JAN}[JAN]{Japanese Article Number}
\acrodef{JDBC}[JDBC]{Java DataBase Connectivity}
\acrodef{JMB}[JMB]{Java--Matlab Bridge}
\acrodef{JPEG}[JPEG]{Joint Photographic Experts Group}
\acrodef{JSON}[JSON]{JavaScript Object Notation}
\acrodef{JSP}[JSP]{Java Server Pages}
\acrodef{JTE}[JTE]{Java-Treceval Engine}
\acrodef{KDE}[KDE]{Kernel Density Estimation}
\acrodef{KLD}[KLD]{Kullback-Leibler Divergence}
\acrodef{KLAPER}[KLAPER]{Kernel LAnguage for PErformance and Reliability analysis}
\acrodef{LAM}[LAM]{Libraries, Archives, and Museums}
\acrodef{LAM2}[LAM]{Logistic Average Misclassification}
\acrodef{LAN}[LAN]{Local Area Network}
\acrodef{LD}[LD]{Linked Data}
\acrodef{LEAF}[LEAF]{Linking and Exploring Authority Files}
\acrodef{LIDO}[LIDO]{Lightweight Information Describing Objects}
\acrodef{LIFO}[LIFO]{Last-In / First-Out}
\acrodef{LM}[LM]{Language Model}
\acrodef{LMT}[LMT]{Log Management Tool}
\acrodef{LOD}[LOD]{Linked Open Data}
\acrodef{LODE}[LODE]{Linking Open Descriptions of Events}
\acrodef{LpO}[LpO]{Leave-$p$-Out}
\acrodef{LRM}[LRM]{Local Relational Model}
\acrodef{LRU}[LRU]{Last Recently Used}
\acrodef{LS}[LS]{Lexical Signature}
\acrodef{LSM}[LSM]{Log Storing Manager}
\acrodef{LtR}[LtR]{Learning to Rank}
\acrodef{LUG}[LUG]{Lexical Unit Generator}
\acrodef{MA}[MA]{Mobile Agent}
\acrodef{MA}[MA]{Moving Average}
\acrodef{MACS}[MACS]{Multilingual ACcess to Subjects}
\acrodef{MADCOW}[MADCOW]{Multimedia Annotation of Digital Content Over the Web}
\acrodef{MAD}[MAD]{Mean Assessed Documents}
\acrodef{MADP}[MADP]{Mean Assessed Documents Precision}
\acrodef{MADS}[MADS]{Metadata Authority Description Standard}
\acrodef{MAP}[MAP]{Mean Average Precision}
\acrodef{MARC}[MARC]{Machine Readable Cataloging}
\acrodef{MATTERS}[MATTERS]{MATlab Toolkit for Evaluation of information Retrieval Systems}
\acrodef{MDA}[MDA]{Model Driven Architecture}
\acrodef{MDD}[MDD]{Model-Driven Development}
\acrodef{METS}[METS]{Metadata Encoding and Transmission Standard}
\acrodef{MIDI}[MIDI]{Musical Instrument Digital Interface}
\acrodef{MIME}[MIME]{Multipurpose Internet Mail Extensions}
\acrodef{ML}[ML]{Machine Learning}
\acrodef{MLE}[MLE]{Maximum Likelihood Estimation}
\acrodef{MLIA}[MLIA]{MultiLingual Information Access}
\acrodef{MM}[MM]{Machinery Model}
\acrodef{MMU}[MMU]{Memory Management Unit}
\acrodef{MODS}[MODS]{Metadata Object Description Schema}
\acrodef{MOF}[MOF]{Meta-Object Facility}
\acrodef{MP}[MP]{Markov Precision}
\acrodef{MPEG}[MPEG]{Motion Picture Experts Group}
\acrodef{MRD}[MRD]{Machine Readable Dictionary}
\acrodef{MRF}[MRF]{Markov Random Field}
\acrodef{MRR}[MRR]{Mean Reciprocal Rank}
\acrodef{MS}[MS]{Mean Squares}
\acrodef{MSAC}[MSAC]{Multilingual Subject Access to Catalogues}
\acrodef{MSE}[MSE]{Mean Square Error}
\acrodef{MT}[MT]{Machine Translation}
\acrodef{MV}[MV]{Majority Vote}
\acrodef{MVC}[MVC]{Model-View-Controller}
\acrodef{NACSIS}[NACSIS]{NAtional Center for Science Information Systems}
\acrodef{NAP}[NAP]{Network processors Applications Profile}
\acrodef{NCP}[NCP]{Normalized Cumulative Precision}
\acrodef{nCG}[nCG]{Normalized Cumulated Gain}
\acrodef{nCRP}[nCRP]{Normalized Cumulated Relative Position}
\acrodef{nDCG}[nDCG]{Normalized Discounted Cumulated Gain}
\acrodef{nMCG}[nMCG]{Normalized Markov Cumulated Gain}
\acrodef{NESTOR}[NESTOR]{NEsted SeTs for Object hieRarchies}
\acrodef{NEXI}[NEXI]{Narrowed Extended XPath I}
\acrodef{NII}[NII]{National Institute of Informatics}
\acrodef{NISO}[NISO]{National Information Standards Organization}
\acrodef{NIST}[NIST]{National Institute of Standards and Technology}
\acrodef{NLP}[NLP]{Natural Language Processing}
\acrodef{NN}[NN]{Neural Network}
\acrodef{NP}[NP]{Network Processor}
\acrodef{NR}[NR]{Normalized Recall}
\acrodef{NRS}[NRS]{Numerical Relational System}
\acrodef{NS-M}[NS-M]{Nested Set Model}
\acrodef{NTCIR}[NTCIR]{NII Testbeds and Community for Information access Research}
\acrodef{OAI}[OAI]{Open Archives Initiative}
\acrodef{OAI-ORE}[OAI-ORE]{Open Archives Initiative Object Reuse and Exchange}
\acrodef{OAI-PMH}[OAI-PMH]{Open Archives Initiative Protocol for Metadata Harvesting}
\acrodef{OAIS}[OAIS]{Open Archival Information System}
\acrodef{OC}[OC]{Operation Code}
\acrodef{OCLC}[OCLC]{Online Computer Library Center}
\acrodef{OMG}[OMG]{Object Management Group}
\acrodef{OO}[OO]{Object Oriented}
\acrodef{OODB}[OODB]{Object-Oriented \acs{DB}}
\acrodef{OODBMS}[OODBMS]{Object-Oriented \acs{DBMS}}
\acrodef{OPAC}[OPAC]{Online Public Access Catalog}
\acrodef{OQL}[OQL]{Object Query Language}
\acrodef{ORP}[ORP]{Open Relevance Project}
\acrodef{OSIRIS}[OSIRIS]{Open Service Infrastructure for Reliable and Integrated process Support}
\acrodef{P}[P]{Precision}
\acrodef{P2P}[P2P]{Peer-To-Peer}
\acrodef{PA}[PA]{Passive Agreements}
\acrodef{PAMT}[PAMT]{Pool-Assessment Management Tool}
\acrodef{PASM}[PASM]{Pool-Assessment Storing Manager}
\acrodef{PC}[PC]{Program Counter}
\acrodef{PCP}[PCP]{Pre-Commercial Procurement}
\acrodef{PCR}[PCR]{Peripherical Command Register}
\acrodef{PD}[PD]{Passive Disagreements}
\acrodef{PDA}[PDA]{Personal Digital Assistant}
\acrodef{PDF}[PDF]{Probability Density Function}
\acrodef{PDR}[PDR]{Peripherical Data Register}
\acrodef{PIR}[PIR]{Personalized Information Retrieval}
\acrodef{POI}[POI]{\acs{PURL}-based Object Identifier}
\acrodef{PoS}[PoS]{Part of Speech}
\acrodef{PAA}[PAA]{Proportion of Active Agreements}
\acrodef{PPA}[PPA]{Proportion of Passive Agreements}
\acrodef{PPE}[PPE]{Programmable Processing Engine}
\acrodef{PREFORMA}[PREFORMA]{PREservation FORMAts for culture information/e-archives}
\acrodef{PRIMAD}[PRIMAD]{Platform, Research goal, Implementation, Method, Actor, and Data}
\acrodef{PRIMAmob-UML}[PRIMAmob-UML]{mobile \acs{PRIMA-UML}}
\acrodef{PRIMA-UML}[PRIMA-UML]{PeRformance IncreMental vAlidation in \acs{UML}}
\acrodef{PROM}[PROM]{Programmable \acs{ROM}}
\acrodef{PROMISE}[PROMISE]{Participative Research labOratory  for Multimedia and Multilingual Information Systems Evaluation}
\acrodef{pSQL}[pSQL]{propagate \acs{SQL}}
\acrodef{PUI}[PUI]{Participant User Interface}
\acrodef{PURL}[PURL]{Persistent \acs{URL}}
\acrodef{QA}[QA]{Question Answering}
\acrodef{QE}[QE]{Query Expansion}
\acrodef{QoS-UML}[QoS-UML]{\acs{UML} Profile for QoS and Fault Tolerance}
\acrodef{QPA}[QPA]{Query Performance Analyzer}
\acrodef{QPP}[QPP]{Query Performance Prediction}
\acrodef{R}[R]{Recall}
\acrodef{RAM}[RAM]{Random Access Memory}
\acrodef{RAMM}[RAM]{Random Access Machine}
\acrodef{RBO}[RBO]{Rank-Biased Overlap}
\acrodef{RBP}[RBP]{Rank-Biased Precision}
\acrodef{RBTO}[RBTO]{Rank-Based Total Order}
\acrodef{RDBMS}[RDBMS]{Relational \acs{DBMS}}
\acrodef{RDF}[RDF]{Resource Description Framework}
\acrodef{REST}[REST]{REpresentational State Transfer}
\acrodef{REV}[REV]{Remote Evaluation}
\acrodef{RF}[RF]{Relevance Feedback}
\acrodef{RFC}[RFC]{Request for Comments}
\acrodef{RIA}[RIA]{Reliable Information Access}
\acrodef{RMSE}[RMSE]{Root Mean Square Error}
\acrodef{RMT}[RMT]{Run Management Tool}
\acrodef{ROM}[ROM]{Read Only Memory}
\acrodef{ROMIP}[ROMIP]{Russian Information Retrieval Evaluation Seminar}
\acrodef{RoMP}[RoMP]{Rankings of Measure Pairs}
\acrodef{RoS}[RoS]{Rankings of Systems}
\acrodef{RP}[RP]{Relative Position}
\acrodef{RR}[RR]{Reciprocal Rank}
\acrodef{RSM}[RSM]{Run Storing Manager}
\acrodef{RST}[RST]{Rhetorical Structure Theory}
\acrodef{RSV}[RSV]{Retrieval Status Value}
\acrodef{RT-UML}[RT-UML]{\acs{UML} Profile for Schedulability, Performance and Time}
\acrodef{SA}[SA]{Software Architecture}
\acrodef{SAL}[SAL]{Storing Abstraction Layer}
\acrodef{SAMT}[SAMT]{Statistical Analysis Management Tool}
\acrodef{SAN}[SAN]{Sistema Archivistico Nazionale}
\acrodef{SASM}[SASM]{Statistical Analysis Storing Manager}
\acrodef{SBTO}[SBTO]{Set-Based Total Order}
\acrodef{SD}[SD]{Sequence Diagram}
\acrodef{SE}[SE]{Search Engine}
\acrodef{SEBD}[SEBD]{Convegno Nazionale su Sistemi Evoluti per Basi di Dati}
\acrodef{SEM}[SEM]{Standard Error of the Mean}
\acrodef{SERP}[SERP]{Search Engine Result Page}
\acrodef{SFT}[SFT]{Satisfaction--Frustration--Total}
\acrodef{SIL}[SIL]{Service Integration Layer}
\acrodef{SIP}[SIP]{Submission Information Package}
\acrodef{SKOS}[SKOS]{Simple Knowledge Organization System}
\acrodef{SM}[SM]{Software Model}
\acrodef{SME}[SME]{Statistics--Metrics-Experiments}
\acrodef{SMART}[SMART]{System for the Mechanical Analysis and Retrieval of Text}
\acrodef{SoA}[SoA]{Service-oriented Architectures}
\acrodef{SOA}[SOA]{Strength of Association}
\acrodef{SOAP}[SOAP]{Simple Object Access Protocol}
\acrodef{SOM}[SOM]{Self-Organizing Map}
\acrodef{SPARQL}[SPARQL]{Simple Protocol and RDF Query Language}
\acrodef{SPE}[SPE]{Software Performance Engineering}
\acrodef{SPINA}[SPINA]{Superimposed Peer Infrastructure for iNformation Access}
\acrodef{SPLIT}[SPLIT]{Stemming Program for Language Independent Tasks}
\acrodef{SPOOL}[SPOOL]{Simultaneous Peripheral Operations On Line}
\acrodef{SQL}[SQL]{Structured Query Language}
\acrodef{SR}[SR]{Sliding Ratio}
\acrodef{sRBP}[sRBP]{Session Rank Biased Precision}
\acrodef{SRU}[SRU]{Search/Retrieve via \acs{URL}}
\acrodef{SS}[SS]{Sum of Squares}
\acrodef{SSD}[s.s.d.]{statistically significantly different}
\acrodef{SSTF}[SSTF]{Shortest Seek Time First}
\acrodef{STAR}[STAR]{Steiner-Tree Approximation in Relationship graphs}
\acrodef{STON}[STON]{STemming ON}
\acrodef{SVM}[SVM]{Support Vector Machine}
\acrodef{TAC}[TAC]{Text Analysis Conference}
\acrodef{TBG}[TBG]{Time-Biased Gain}
\acrodef{TCP}[TCP]{Transmission Control Protocol}
\acrodef{TEL}[TEL]{The European Library}
\acrodef{TERRIER}[TERRIER]{TERabyte RetrIEveR}
\acrodef{TF}[TF]{Term Frequency}
\acrodef{TFR}[TFR]{True False Rate}
\acrodef{TLD}[TLD]{Top Level Domain}
\acrodef{TME}[TME]{Topics--Metrics-Experiments}
\acrodef{TN}[TN]{True Negative}
\acrodef{TO}[TO]{Transfer Object}
\acrodef{TP}[TP]{True Positve}
\acrodef{TPR}[TPR]{True Positive Rate}
\acrodef{TRAT}[TRAT]{Text Relevance Assessing Task}
\acrodef{TREC}[TREC]{Text REtrieval Conference}
\acrodef{TRECVID}[TRECVID]{TREC Video Retrieval Evaluation}
\acrodef{TTL}[TTL]{Time-To-Live}
\acrodef{UCD}[UCD]{Use Case Diagram}
\acrodef{UDC}[UDC]{Universal Decimal Classification}
\acrodef{uGAP}[uGAP]{User-oriented Graded Average Precision}
\acrodef{UI}[UI]{User Interface}
\acrodef{UML}[UML]{Unified Modeling Language}
\acrodef{UMT}[UMT]{User Management Tool}
\acrodef{UMTS}[UMTS]{Universal Mobile Telecommunication System}
\acrodef{UoM}[UoM]{Utility-oriented Measurement}
\acrodef{UPC}[UPC]{Universal Product Code}
\acrodef{URI}[URI]{Uniform Resource Identifier}
\acrodef{URL}[URL]{Uniform Resource Locator}
\acrodef{URN}[URN]{Uniform Resource Name}
\acrodef{USM}[USM]{User Storing Manager}
\acrodef{VA}[VA]{Visual Analytics}
\acrodef{VAIRE}[VAIR\"{E}]{Visual Analytics for Information Retrieval Evaluation}
\acrodef{VATE}[VATE$^2$]{Visual Analytics Tool for Experimental Evaluation}
\acrodef{VIRTUE}[VIRTUE]{Visual Information Retrieval Tool for Upfront Evaluation}
\acrodef{VD}[VD]{Virtual Document}
\acrodef{VDM}[VDM]{Visual Data Mining}
\acrodef{VIAF}[VIAF]{Virtual International Authority File}
\acrodef{VIM}[VIM]{International Vocabulary of Metrology}
\acrodef{VL}[VL]{Visual Language}
\acrodef{VoIP}[VoIP]{Voice over IP}
\acrodef{VS}[VS]{Visual Sentence}
\acrodef{W3C}[W3C]{World Wide Web Consortium}
\acrodef{WAN}[WAN]{Wide Area Network}
\acrodef{WHO}[WHO]{World Health Organization}
\acrodef{WLAN}[WLAN]{Wireless \acs{LAN}}
\acrodef{WP}[WP]{Work Package}
\acrodef{WS}[WS]{Web Services}
\acrodef{WSD}[WSD]{Word Sense Disambiguation}
\acrodef{WSDL}[WSDL]{Web Services Description Language}
\acrodef{WWW}[WWW]{World Wide Web}
\acrodef{XMI}[XMI]{\acs{XML} Metadata Interchange}
\acrodef{XML}[XML]{eXtensible Markup Language}
\acrodef{XPath}[XPath]{XML Path Language}
\acrodef{XSL}[XSL]{eXtensible Stylesheet Language}
\acrodef{XSL-FO}[XSL-FO]{\acs{XSL} Formatting Objects}
\acrodef{XSLT}[XSLT]{\acs{XSL} Transformations}
\acrodef{YAGO}[YAGO]{Yet Another Great Ontology}
\acrodef{YASS}[YASS]{Yet Another Suffix Stripper}



\end{document}
